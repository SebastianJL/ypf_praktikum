\documentclass[10pt,a4paper]{article}

\usepackage[utf8]{inputenc}
\usepackage{graphicx}
\usepackage{gensymb}
\usepackage{amsmath}
\usepackage{amssymb}
\usepackage{geometry}
\usepackage{multicol}  
\usepackage{siunitx}
\usepackage{float}
\usepackage{color}
\graphicspath{{figures/}}

\geometry{a4paper}

\author{Nik Dennler \\ Johannes Lade}
\title{Digital Electronics Lab}
\date{\today{}}


\begin{document}
	
\begin{titlepage}
	\maketitle
		\begin{center}
			Email: johannes.lade@uzh.ch
		\end{center}
	\thispagestyle{empty}
\end{titlepage}

\tableofcontents
\newpage


\section{Know Errors}

\section{Exercise Block 1}

\subsection{Exclusive OR (15min)}\label{subsec:ex-1}
\textbf{Construct an XOR gate using only the giant NAND gates.}

 XOR stands for exclusive OR. As the name suggest, it is very similar to the OR gate, except that it does not yield one when both inputs are one. In different words this means that the XOR is true whenever the two inputs are not equal. Follow the steps below to construct the XOR circuit.
\begin{enumerate}
	\item Write down the truth table.
	\item\label{it:1} Construct the disjunctive normal form from your truth table.
	\item Draw the circuit according to step \ref{it:1}.
	\item Transform the circuit in such a way that it only uses NAND Gates.
	\item Build the XOR you have drawn with the giant NAND gates.
	\item How many NAND gates do you need? See if you can loose one or two gates.
\end{enumerate}

\subsection{XOR to the Next (15min)}
\textbf{Construct the same XOR gate as in exercise \ref{subsec:ex-1} but this time with the breadboard using the tiny gates.}


\section{Excercise Block 2}

\subsection{Half-Adder (30min)}
\textbf{Build a half-adder on the bread board.}

A half-adder takes two bits $a$ and $b$ and adds them. It is called half-adder because two of them are needed in order to construct an adder which is able two add several bits. We will see later why this is the case.
\begin{table}[H]
	\centering
	\begin{tabular}{|c|c||c|c|}
		\hline
		$a$ & $b$ & $c_{out}$ & $out$ \\ \hline
		&     &           &       \\ \hline
		&     &           &       \\ \hline
		&     &           &       \\ \hline
		&     &           &       \\ \hline
	\end{tabular}
	\caption{Truth table for the Half Adder.}
	\label{tab:half-adder-truth-table}
\end{table}

\subsection{Full-Adder (30min)}

\section{Known Errors}
\subsection{Giant Logic Gates}
\begin{itemize}
	\item Check wether the power supply has a loose connection.
\end{itemize}
\subsection{Bread Board}
\begin{itemize}
	\item \textbf{Chip gets hot.} 
	\begin{itemize}
		\item You connected the power supply in the wrong direction. The chip consists of transistors i.e. diods. This means it only allows for the current to pass through the chip in one direction. Otherwise it gets grilled. Ask an assistant you'll probably have to throw it away.
	\end{itemize}
	\item \textbf{The led is not glowing.}
	\begin{itemize}
		\item Did you use the led we provided on the breadboard? It is set-up such that you don't have to think about how you have to connect it.
		\item Should it really light up? Check with the voltmeter if your gate output really is at logic one.
		\item If you really want to connect your own led think about two things. First you need to connect a $3\si{\kilo\ohm}$ resistor in series. Otherwise the led will blow. Second, remember that led stands for Light Emitting Diod. This means that you have to connect it with the longer leg towards the high voltage. 
	\end{itemize}
	\item \textbf{You forgott to connect the power supply.}
	\begin{itemize}
		\item Well I guess you know what you have to do now ;)
	\end{itemize}
	\item \textbf{Your measurements on the gate ouput don't fit your expectations.}
	\begin{itemize}
		\item What did you measure? Remember that the logic follows the voltage. In the range of $0-0.8\si{\volt}$ the output is on logic $0$. In the range of $2-5\si{\volt}$ the output is on logic $1$. The current is not affected by this and doesn't have to follow the same principle.
	\end{itemize}
\end{itemize}
\end{document}



