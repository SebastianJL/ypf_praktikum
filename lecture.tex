\documentclass[10pt,a4paper]{article}

\usepackage[utf8]{inputenc}
\usepackage{graphicx}
\usepackage{gensymb}
\usepackage{amsmath}
\usepackage{amssymb}
\usepackage{geometry}
\usepackage{multicol}  
\usepackage{siunitx}
\usepackage{float}
\usepackage{color}
\graphicspath{{figures/}}

\geometry{a4paper}

\author{Nik Dennler \\ Johannes Lade}
\title{Digital Electronics Lab}
\date{\today{}}


\begin{document}
	
\begin{titlepage}
	\maketitle
		\begin{center}
			Email: nik.dennler@uzh.ch, johannes.lade@uzh.ch
		\end{center}
	\thispagestyle{empty}
\end{titlepage}




\tableofcontents
\newpage

\section{Catchy Introduction}
\subsection{How to build a computer?}
What can a computer do?
Add
Save

Big question is actually how to handle numbers. Difficult for computer/circuits to differ between range of voltages, easiest way is just to have two states. So our conventional system (decimal) is not very convenient, would be easier to have system that works with two numbers only. 





\section{Quick Theory}
\subsection{Digital Circuits}
Talking of circuits, we differ between analogue circuits and digital circuits. The difference lays actually in the format of the signal that represents the carried information. \newline
An analogue signal uses some attribute of the medium to convey the signal's information. In electric circuits is it the voltage, current or frequency, that is varied in order to create an electric signal. Analog electronic circuits are those in which the threshold of the current or voltage varies \underline{continuously} over time, corresponding to the information being represented. They follow Kirchhoff's circuit laws: all the currents at a node (a place where wires meet), and the voltage around a closed loop of wires is zero. \newline
In digital electronic circuits, electric signals take on \underline{discrete values} to represent logical and numeric values, which then represent the information that is being processed. Every digital circuit is also a analog circuit, but it has ideally (just two) discrete potential levels in use. 

\subsection{Binary Numbers}
%erkläre mehr, über Dezimalzahlsystem, Vergleich Dezimal/Binär, Übertrag, etc.!!! Motivation, wir haben nur zwei Zustände. 
Almost always, binary encoding is used: one voltage represents a binary '1' and another voltage, usually a value near the ground potential, represents a binary '0'. Those binary digits, each of them called one 'bit', can then be used to represent numbers and characters and therefore to carry information. The binary code was invented and first used by Gottfried Leibniz in 1679.

\begin{figure}[H]
\centering
\begin{minipage}{0.5\textwidth}
%  \centering
  \includegraphics[width=1\textwidth]{binary.png}%
  \caption{Binary representation of one bit}%
  \label{fig:binary}
\end{minipage}%
\begin{minipage}{.4\textwidth}
  \centering
  \includegraphics[width=0.8\textwidth]{leibniz}%
  \caption{Painting of Gottfried Leibniz}%
  \label{fig:leibniz}
\end{minipage}
\end{figure}

\subsection{Boolean Algebra and its implementation}
To understand how a processor can actually perform calculations and follow algorithms with those binary numbers, some knowledge about Boolean logic has to be obtained first. The Boolean Algebra (first half of the 19. century by George Boole) defines three main operations: 

%\noindent
\begin{itemize}
 \item \textbf{The conjunction:} $a\land b$ is $1$ if and only if $a$ is $1$ and $b$ is $1$.\\
 \item \textbf{The disjunction:} $a\lor b$  is $1$ if and only if $a$ is $1$ or $b$ is $1$ or $a\wedge b$ is $1$.\\
 \item \textbf{The negation:}  $\neg a$ is $1$ if and only if $a$ is $0$. A negated variable $a$ is also often written as $\overline{a}$.
\end{itemize}

\noindent
The operations of the Boolean Algebra are Associative, Commutative and Distributive. They have a Neutral, an Inverse and a Zero element. One can easily imagine, that those operations can be combined and added in number to create more complex calculation operators. This is actually exactly what happens in our computers, servers and smartphones, of course on a very large scale. 
\subsection{Truth tables}
These Boolean operators can be represented in so called truth tables as follows:

%%include also ==, XOR, etc... 

\begin{table}[H]
\centering
\begin{tabular}{c|c|c|c|c}
\textbf{$a$} & \textbf{$b$} & \textbf{$a\land b$} & \textbf{$a\lor b$} & \textbf{$\neg a$} \\ \hline
0          & 0          & 0            & 0            & 1           \\
1          & 0          & 0            & 1            & 0           \\
0          & 1          & 0            & 1            & 1           \\
1          & 1          & 1            & 1            & 0          
\end{tabular}
\caption{truth table}
\label{tab:truth}
\end{table}

One can also represent more operations than the three basic one. Let's look for example at the comparison operator, the exclusive or and the negated and. 


\subsection{Logical Formula in Disjunctive normal-form}
To construct a function, one wants to convert the information in the truth table in one compact formula. We use the disjuctive normal-form, which works as follows: 
\newline \textbf{In the column of the operator output, look for logical $1$. Now for each line that has a $1$ in the output field, combine the input variables with a logical \textit{and} ($\land$). If the variable has a entry of $0$, we use the inversion notation with an overline. Then connect those expressions with a logical \textit{or} ($\lor$).} 
\newline Now we have the boolean function in \underline{disjunctive normal-form} that describes our operator. Let's make two examples: \newline
\begin{enumerate}
 \item [\textbf{AND:}]$(a\land b)$
 \item [\textbf{OR:}]$(a\land \overline{b})\lor(\overline{a}\land b)\lor(a\land b)$
\end{enumerate}

As you can see, does the expression for the \textbf{OR} look unnecessarily complicated, we could just write $(a\lor b)$. This would be in conjunctive normal-form, but the point is, to use always the same logical system (disjunctive normal-form!). When we have described the operator in this form, we can built it up easily with basic circuits elements.  

\subsection{Drawing of a circuit}
After having a function that describes our operator in disjunctive normal-form, is it very easy to translate this into a circuit. We just replace the symbols in the fomula with the circuit symbols that are shown here. For larger circuits do we then connect all the inputs and outputs according to the structure (parenthesis!) of the formula.

\begin{figure}[H]
\centering
  \includegraphics[height=0.8\textwidth]{logic_symbols}%
  \caption{Logic circuit symbols}%
  \label{fig:logic_symbols}
\end{figure}
Once drawn, such a circuit can look like this:

\begin{figure}[H]
\centering
  \includegraphics[height=0.3\textwidth]{equivalence}%
  \caption{Example for circuit design: the equivalence}%
  \label{fig:equivalence}
\end{figure}

\noindent
Now, before we set up this circuit, we have to think practical: Would it not be much more beautiful, and also much more easy, to just have one element instead of three different ones? In electronics, especially today in microelectronics (computers!), it is a key factor to build a circuit as easy as possible. Now, one can prove mathematically, that every circuit that is build up by OR's, AND's and Inversions, can also be build up by NAND's (negated AND) as well as by NOR's (negated OR). The circuit is shown again, but this time the elements were replaced by NAND's. 

\begin{figure}[H]
\centering
  \includegraphics[height=0.2\textwidth]{equivalence_optimized}%
  \caption{Optimized equivalence circuit}%
  \label{fig:equivalence_optimized}
\end{figure}

So how do we actually replace the element with NAND's? To figure out the conversion, we compare the truth table of the NAND with that of different conminations of the three basic elements. We see, that a NAND with inverted output is a NAND (per definition) and an OR with inverted inputs is also a NAND. The conversions are seen below again. The clue is now, that if we insert anywhere two inverters after each other, it does not change the signal. After doing this, one can use the conversions below and end up with a circuit completely build up of NAND's. 

\begin{figure}[H]
\centering
  \includegraphics[height=0.33\textwidth]{nand_conversion}%
  \caption{Conversion to NAND}%
  \label{fig:nand_conversion}
\end{figure}

\section{Combinatoric circuits}
\subsection{General}
We differ between combinatoric circuits and sequential circuits. Combinatoric circuits are logical functions, whose \underline{output states are always exactly defined through their input variables}. This point will become more clear later when we look at sequential circuits. 

\subsection{Construction of a combinatoric circuit}
\begin{enumerate}
	\item Write down the truth table.
	\item Construct the disjunctive normal form from your truth table.
	\item Draw the circuit accordingly.
	\item Transform the circuit in such a way that it only uses NAND Gates.
	      How many NAND gates do you need? See if you can loose one or two gates.
	\item Check/discuss the circuit design again. An effective method is to create the truth table again just by looking at the circuit, then compare with what you should get. 
	\item If you're sure that your design works, build up the circuit with real components and connections. 
\end{enumerate}



%\subsubsection{Truth table}                            
%\subsubsection{Equation}                               
%\subsubsection{Draw circuit with logic elements}       
%\subsubsection{Unify}                                  
                                                       
\section{Sequential circuit}                           
\subsection{General}
Now we look at sequential circuits, which is different from the combinatoric circuit. Sequential circuits have something that can be looked at as a memory. The clue is, that the output of an operator is used again as one of the inputs. This means, that the circuit depends not only on combinatoric input variables, but also from the output of the step before. In this way, the functionality of a memory is realized. 

\subsection{Elements}
\subsubsection{Latch}
This concept is easily explained with the most basic realization of a sequential circuit. If the output has a default value of zero (this means, that there is no current on the output line in the beginning), then this zero-value also goes to one of the inputs. Now this stays as it is as long \texttt{A=0} as well. But now, if for once \texttt{A=1}, the OR-Operator will bring an output of \texttt{1}. Now, even if \texttt{A=0} again, will the output stay on \texttt{1}. We can say, that we saved the binary number \texttt{1}.

\begin{figure}[H]
\centering
  \includegraphics[height=0.2\textwidth]{loop}%
  \caption{Latch}%
  \label{fig:loop}
\end{figure}

Of course, this is not very sophisticated yet, we cannot remove the output again or save something else. 

\subsubsection{Clock}
\subsection{Construction of a sequential circuit}

\newpage
\section{Exercise Block 1}
\label{sec:exercise-block-1}

\subsection{Exclusive OR (15min)}\label{subsec:ex-1}
\textbf{Construct an XOR gate using only the giant NAND gates.}

 XOR stands for exclusive OR. As the name suggest, it is very similar to the OR gate, except that it does not yield one when both inputs are one. In different words this means that the XOR is true whenever the two inputs are not equal. Follow the steps below to construct the XOR circuit.
\begin{enumerate}
	\item Write down the truth table.
	\item\label{it:1} Construct the disjunctive normal form from your truth table.
	\item Draw the circuit according to step \ref{it:1}.
	\item Transform the circuit in such a way that it only uses NAND Gates.
	\item Build the XOR you have drawn with the giant NAND gates.
	\item How many NAND gates do you need? See if you can loose one or two gates.
\end{enumerate}

\subsection{XOR to the Next (15min)}
\textbf{Construct the same XOR gate as in exercise \ref{subsec:ex-1} but this time with the breadboard using the tiny gates.}


\section{Excercise Block 2a}

\subsection{Half-Adder (30min)}
\subsubsection{Part 1}
A half-adder takes two bits $a$ and $b$ and adds them. It is called half-adder because two of them are needed in order to construct an adder which is able to add several bits. We will see later why this is the case.

Let's first look at the addition of two decimal digits. 

\begin{figure}[h]
	\centering		  
	\includegraphics[scale=0.3]{decimal_addition.png}
	\caption{Decimal addition.}
	\label{fig:decimal_addition}
\end{figure}
We see that we need two things:
\begin{itemize}
	\item We need to know in which number the combination of 5 and 9 results. Namely 4.
	\item We need to carry a digit 1 because $5+9$ exceeds 10.
\end{itemize}

For binary digits (bits) we can do the same thing. Let's assume the following case. We have two 1 digit binary numbers. We call the single bit of the first number $a$ and the single bit of the second number $b$.
\begin{itemize}
	\item Write down all possible combinations of $a$ and $b$ in table \ref{tab:half-adder-truth-table}.
	\item Think about the output of each combination of $a$ and $b$ and write it into table \ref{tab:half-adder-truth-table} in the column $\Sigma$ (for sum). Which logical circuit does this resemble?
	\item Similar to the decimal case, we need to sometimes carry a bit. Write the value of the carry bit $c$ into table \ref{tab:half-adder-truth-table}. Now look at the columns $a$, $b$ and $c$. Which logical circuit does this resemble?
	
	\begin{table}[H]
		\centering
		\begin{tabular}{|c|c||c|c|}
			\hline
			$a$ & $b$ & $\Sigma$   & $c$ \\ \hline
			&     &           &       \\ \hline
			&     &           &       \\ \hline
			&     &           &       \\ \hline
			&     &           &       \\ \hline
		\end{tabular}
		\caption{Truth table for the Half Adder.}
		\label{tab:half-adder-truth-table}
	\end{table}

	\item Draw the two circuits from table \ref{tab:half-adder-truth-table} with the same steps as in section \ref{subsec:ex-1} without actually building them. (Hint: One of the two circuits you know already. Use the reduced form from the last exercise.)
	\item With the two ciruits drawn entirely with NAND gates, see if you can combine them into one.
	\item Congrats, you have successfully drawn the circuit of a half adder. Now build one on the breadboard.
\end{itemize}
\subsection{Full-Adder (30min)}
You may have noticed that the half-adder - although you can add two bits and generate a carry bit - is not enough to add several bits together. Let's see why.

If we look now at two 2 bit numbers: $a_1a_0$ and $b_1b_0$. Then the half-adder is enough in order to add the first two bits $a_0$ and $b_0$. But as soon as we want to add the next two bits $a_1$ and $b_1$ we also need to add the carry bit from the first operation. So we have $a_1+b_1+c_1$, but the half-adder is not capable of adding three bits. Consequently, if we were to build a two bit adder only with half-adders then the carry bits would just be ignored. This would lead to wrong results like this:
\[
	01_2 + 11_2 = 10_2
\]
or in decimal
\[
	1_{10} + 3_{10} = 2_{10}
\]
Lukily the full-adder - which can add three bits - is easily built with two half-adders, hence the name. As the derivation of the full adder is beyond the scope of this lecture, we will simply introduce its logic circuit. For simplicity we pack the circuit of the half adder into one black box and simply denote its outputs with $\Sigma$ and $C$ (See figure \ref{fig:full-adder}).

\begin{figure}[h]
	\centering		  
	\includegraphics[scale=0.3]{full_adder.png}
	\caption{Full adder built of two half-adders.}
	\label{fig:full-adder}
\end{figure}
\begin{itemize}
	\item Fill in all possible combinations for
\end{itemize}
\begin{table}[]
	\centering
	\begin{tabular}{|c|c|c||c|c|}
		\hline
		$c_{in}$ & $a$ & $b$ & $\Sigma$ & $c_{out}$ \\ \hline
		&     &     &          &           \\ \hline
		&     &     &          &           \\ \hline
		&     &     &          &           \\ \hline
		&     &     &          &           \\ \hline
		&     &     &          &           \\ \hline
		&     &     &          &           \\ \hline
		&     &     &          &           \\ \hline
		&     &     &          &           \\ \hline
	\end{tabular}
	\caption{Truth table for the Full Adder.}
	\label{tab:full-adder-truth-table}
\end{table}

\section{Exercise Block 2b}
\subsection{D FlipFlop}
\subsection{D Master Slave}

\newpage
\section{Exercise Skeleton}
\begin{itemize}
	\item Exercise Block 1
	\begin{itemize}
		\item Exercise: Exclusive Or.
		\item Short discussion. Show smallest circuit.
		\item Exercise: XOR to the Next.
		\item Short discussion. Problems etc
	\end{itemize}
	\item Split groups
	\item Exercise Block 2a
	\begin{itemize}
		\item 
	\end{itemize}
	\item Exercise Block 2b
\end{itemize}
\newpage
\section{Known Errors}
\subsection{Giant Logic Gates}
\begin{itemize}
	\item Check wether the power supply has a loose connection.
\end{itemize}
\subsection{Bread Board}
\begin{itemize}
	\item \textbf{Chip gets hot.} 
	\begin{itemize}
		\item You connected the power supply in the wrong direction. The chip consists of transistors i.e. diods. This means it only allows for the current to pass through the chip in one direction. Otherwise it gets grilled. Ask an assistant you'll probably have to throw it away.
	\end{itemize}
	\item \textbf{The led is not glowing.}
	\begin{itemize}
		\item Did you use the led we provided on the breadboard? It is set-up such that you don't have to think about how you have to connect it.
		\item Should it really light up? Check with the voltmeter if your gate output really is at logic one.
		\item If you really want to connect your own led think about two things. First you need to connect a $3\si{\kilo\ohm}$ resistor in series. Otherwise the led will blow. Second, remember that led stands for Light Emitting Diod. This means that you have to connect it with the longer leg towards the high voltage. 
	\end{itemize}
	\item \textbf{You forgott to connect the power supply.}
	\begin{itemize}
		\item Well I guess you know what you have to do now ;)
	\end{itemize}
	\item \textbf{Your measurements on the gate ouput don't fit your expectations.}
	\begin{itemize}
		\item What did you measure? Remember that the logic follows the voltage. In the range of $0-0.8\si{\volt}$ the output is on logic $0$. In the range of $2-5\si{\volt}$ the output is on logic $1$. The current is not affected by this and doesn't have to follow the same principle.
	\end{itemize}
\end{itemize}




\end{document}



